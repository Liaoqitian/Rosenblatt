\documentclass[11pt]{article}
\usepackage{amsmath} 
\usepackage{graphicx}
\usepackage{subcaption}
\usepackage{sectsty}
\usepackage{amssymb}
 \usepackage{lipsum}
\usepackage{titlesec}
\usepackage{romannum}
\usepackage{enumitem}
\usepackage{mathtools}
\usepackage[super]{nth}
\usepackage{tikz}
\usepackage{listings}
\usepackage{pagecolor,lipsum}
\usepackage{color,soul}
\usepackage{xcolor}
\usepackage[T1]{fontenc}
\usepackage{textcomp}
\usepackage{float}
\definecolor{dkgreen}{rgb}{0,0.6,0}
\definecolor{gray}{rgb}{0.5,0.5,0.5}
\definecolor{mauve}{rgb}{0.58,0,0.82}
\definecolor{codegreen}{rgb}{0,0.6,0}
\definecolor{codegray}{rgb}{0.5,0.5,0.5}
\definecolor{codepurple}{rgb}{0.58,0,0.82}
\definecolor{backcolour}{rgb}{0.95,0.95,0.92}
\definecolor{orange}{RGB}{255,127,0}
\pagecolor{white}
\graphicspath{ {./images/} }

\lstdefinelanguage{JavaScript}{
  keywords={typeof, new, catch, function, return, null, catch, switch, var, if, in, while, do, else, case, break, let, log, const, console},
  keywordstyle=\color{blue}\bfseries,
  ndkeywords={class, export, boolean, throw, implements, import, this, true, false},
  ndkeywordstyle=\color{darkgray}\bfseries,
  identifierstyle=\color{black},
  sensitive=false,
  comment=[l]{//},
  morecomment=[s]{/*}{*/},
  commentstyle=\color{purple}\ttfamily,
  stringstyle=\color{red}\ttfamily,
  morestring=[b]',
  morestring=[b]"
}

\lstset{
   %frame = tb
   language=JavaScript,
   backgroundcolor=\color{backcolour},
   extendedchars=true,
   basicstyle=\small\ttfamily,
   showstringspaces=false,
   showspaces=false,
   numberstyle=\tiny\color{codegray},
   ndkeywordstyle=\color{codegreen}\bfseries,
   keywordstyle=\color{blue},
   commentstyle=\color{gray},
   stringstyle=\color{mauve},
   numbers=none,
   %numberstyle=\footnotesize,
   numbersep=9pt,
   tabsize=2,
   breaklines=true,
   showtabs=false,
   captionpos=b, 
   escapeinside={(*@}{@*)} 
}
\iffalse

\lstset{frame=tb,
  language=Javascript,
  aboveskip=3mm,
  belowskip=3mm,
  showstringspaces=false,
  columns=flexible,
  basicstyle={\small\ttfamily},
  numbers=left, %eller none
  numberstyle=\tiny\color{codegray},
  keywordstyle=\color{mauve},
  commentstyle=\color{gray},
  stringstyle=\color{orange},
  breaklines=false,
  breakatwhitespace=true,
  tabsize=3,
  escapeinside={(*@}{@*)}, 
 % backgroundcolor=\color{backcolour},  
}

\fi
\newcommand*\circled[1]{\tikz[baseline=(char.base)]{
            \node[shape=circle,draw,inner sep=2pt] (char) {#1};}}
\setlength\parindent{0pt}
\setlist[itemize,1]{leftmargin=\dimexpr 26pt-.5in}

\sectionfont{\fontsize{12}{15}\selectfont}
\title{Introduction to Programming with Javascript}
\author{Qitian Liao}
\date{Aug 3, 2020} 
\usepackage[left=2cm, right=2cm, top=2cm]{geometry}
%\setlength\parindent{0pt}

\DeclarePairedDelimiter\abs{\lvert}{\rvert}
\DeclarePairedDelimiter\norm{\lVert}{\rVert}

\begin{document}
\begin{titlepage}
	\begin{center} 
	\line(1, 0){400}\\
	[0.25in]
	\huge{\bfseries Introduction to Javascript} \\
	[2mm]
	\line(1, 0){300} \\
	[1.5cm]
	\textsc{\LARGE Qitian Liao} \\
	[0.5cm]
	\textsc{\large University of California, Berkeley} \\
	[15cm]
	\end{center}
	\begin{flushright}	
	\end{flushright}
\end{titlepage}

\thispagestyle{empty}
\newpage
\tableofcontents
\thispagestyle{empty}
\cleardoublepage
\setcounter{page}{1}
\def\Arg{\mathop{\operator@font Arg}\nolimits}
%\newpage
\pagenumbering{arabic}
\titleformat*{\section}{\Large\bfseries}
\titleformat*{\subsection}{\large\bfseries}
\titleformat*{\subsubsection}{\normalsize\bfseries}
\titleformat*{\paragraph}{\large\bfseries}
\titleformat*{\subparagraph}{\large\bfseries}

%\titlespacing\section{0pt}{5pt plus 4pt minus 2pt}{5pt plus 2pt minus 2pt}
%\titlespacing\subsection{0pt}{10pt plus 4pt minus 2pt}{5pt plus 2pt minus 2pt}
%\titlespacing\subsubsection{0pt}{5pt plus 4pt minus 2pt}{5pt plus 2pt minus 2pt}
\newpage
\section{Introduction to Javascript}
\subsection{What is Javascript? }
Last year, millions of learners from our community started with JavaScript. Why? JavaScript is primarily known as the language of most modern web browsers, and its early quirks gave it a bit of a bad reputation. However, the language has continued to evolve and improve. JavaScript is a powerful, flexible, and fast programming language now being used for increasingly complex web development and beyond! \\
\newline
Since JavaScript remains at the core of web development, it’s often the first language learned by self-taught coders eager to learn and build. We’re excited for what you’ll be able to create with the JavaScript foundation you gain here. JavaScript powers the dynamic behavior on most websites, including this one. \\
\newline
In this lesson, you will learn introductory coding concepts including data types and built-in objects—essential knowledge for all aspiring developers. Make sure to take notes and pace yourself. This foundation will set you up for understanding the more complex concepts you’ll encounter later. 
\subsection{Console}
The console is a panel that displays important messages, like errors, for developers. Much of the work the computer does with our code is invisible to us by default. If we want to see things appear on our screen, we can print, or log, to our console directly. \\
\newline
In JavaScript, the \colorbox{lightgray}{console} keyword refers to an object, a collection of data and actions, that we can use in our code. Keywords are words that are built into the JavaScript language, so the computer will recognize them and treats them specially. \\
\newline
One action, or method, that is built into the \colorbox{lightgray}{console} object is the \colorbox{lightgray}{.log()} method. When we write \colorbox{lightgray}{console.log()} what we put inside the parentheses will get printed, or logged, to the console. It is going to be very useful for us to print values to the console, so we can see the work that we are doing.
\begin{lstlisting}
console.log(5); 
\end{lstlisting}
This example logs 5 to the console. The semicolon denotes the end of the line, or statement. Although in JavaScript your code will usually run as intended without a semicolon, we recommend learning the habit of ending each statement with a semicolon so you never leave one out in the few instances when they are required. \\
\newline
You’ll see later on that we can use \colorbox{lightgray}{console.log()} to print different kinds of data.
\subsection{Comments}
Programming is often highly collaborative. In addition, our own code can quickly become difficult to understand when we return to it. For these reasons, it’s often useful to leave notes in our code for other developers or ourselves.\\
\newline
As we write JavaScript, we can write comments in our code that the computer will ignore as our program runs. These comments exist just for human readers. Comments can explain what the code is doing, leave instructions for developers using the code, or add any other useful annotations. There are two types of code comments in JavaScript: 
\begin{enumerate}[leftmargin = *]
\item A \textit{single line comment} will comment out a single line and is denoted with two forward slashes \colorbox{lightgray}{//} preceding it.
\begin{lstlisting} 
// Prints 5 to the console
console.log(5);
\end{lstlisting}
You can also use a single line comment to comment after a line of code: 
\begin{lstlisting}
console.log(5);  // Prints 5 
\end{lstlisting}
\item A \textit{multi-line comment} will comment out multiple lines and is denoted with \colorbox{lightgray}{/*} to begin the comment, and \colorbox{lightgray}{*/} to end the comment.
\begin{lstlisting}
/*
This is all commented 
console.log(10);
None of this is going to run!
console.log(99);
*/
\end{lstlisting} 
You can also use this syntax to comment something out in the middle of a line of code: 
\begin{lstlisting} 
console.log(/*IGNORED!*/ 5);  // Still just prints 5 
\end{lstlisting}
\end{enumerate}
\subsection{Data Types}
\textit{Data types} are the classifications we give to the different kinds of data that we use in programming. In JavaScript, there are seven fundamental data types:
\begin{enumerate}[leftmargin = *]
\item \textit{Number}: Any number, including numbers with decimals: \colorbox{lightgray}{4}, \colorbox{lightgray}{8}, \colorbox{lightgray}{1516}, \colorbox{lightgray}{23.42}.
\item \textit{String}: Any grouping of characters on your keyboard (letters, numbers, spaces, symbols, etc.) surrounded by single quotes: \colorbox{lightgray}{` ... '} or double quotes \colorbox{lightgray}{`` ... "}. Though we prefer single quotes. Some people like to think of string as a fancy word for text. 
\item \textit{Boolean}: This data type only has two possible values— either \colorbox{lightgray}{true} or \colorbox{lightgray}{false} (without quotes). It’s helpful to think of booleans as on and off switches or as the answers to a “yes” or “no” question. 
\item \textit{Null}: This data type represents the intentional absence of a value, and is represented by the keyword \colorbox{lightgray}{null} (without quotes).
\item \textit{Undefined}: This data type is denoted by the keyword \colorbox{lightgray}{undefined} (without quotes). It also represents the absence of a value though it has a different use than \colorbox{lightgray}{null}.
\item \textit{Symbol}: A newer feature to the language, symbols are unique identifiers, useful in more complex coding. No need to worry about these for now.
\item \textit{Object}: Collections of related data.
\end{enumerate}
The first 6 of those types are considered \textit{primitive data types}. They are the most basic data types in the language. \textit{Objects} are more complex, and you will learn much more about them as you progress through JavaScript. At first, seven types may not seem like that many, but soon you will observe the world opens with possibilities once you start leveraging each one. As you learn more about objects, you’ll be able to create complex collections of data. But before we do that, let us first get comfortable with strings and numbers!
\begin{lstlisting}
console.log("Location of The Metropolitan: 2301 Durant Ave, Berkeley");
console.log(40);
\end{lstlisting}
In the example above, we first printed a string. Our string is not just a single word; it includes both capital and lowercase letters, spaces, and punctuation. Next, we printed the number 40, notice we did not use quotes.

\subsection{Arithmetic Operators}
Basic arithmetic often comes in handy when programming. An \textit{operator} is a character that performs a task in our code. JavaScript has several built-in in \textit{arithmetic operators}, that allow us to perform mathematical calculations on numbers. These include the following operators and their corresponding symbols:
\begin{enumerate}[leftmargin = *]
\item Add: \colorbox{lightgray}{$+$} 
\item Subtract: \colorbox{lightgray}{$-$}
\item Multiply: \colorbox{lightgray}{$*$}
\item Divide: \colorbox{lightgray}{$/$}
\item Remainder: \colorbox{lightgray}{$\%$}
\end{enumerate}
The first four work how you might guess:
\begin{lstlisting}
console.log(3 + 4); // Prints 7
console.log(5 - 1); // Prints 4
console.log(4 * 2); // Prints 8
console.log(9 / 3); // Prints 3
\end{lstlisting}
Note that when we \colorbox{lightgray}{console.log()} the computer will evaluate the expression inside the parentheses and print that result to the console. If we wanted to print the characters \colorbox{lightgray}{3 + 4}, we would wrap them in quotes and print them as a string.
\begin{lstlisting}
console.log(11 % 3); // Prints 2
console.log(12 % 3); // Prints 0 
\end{lstlisting}
The remainder operator, sometimes called \textit{modulo}, returns the number that remains after the right-hand number divides into the left-hand number as many times as it evenly can: \colorbox{lightgray}{11 \% 3} equals 2 because 3 fits into 11 three times, leaving 2 as the remainder.
\subsection{String Concatenation}
Operators are not just for numbers! When a \colorbox{lightgray}{$+$} operator is used on two strings, it appends the right string to the left string:
\begin{lstlisting}
console.log("hi" + "ya"); // Prints "hiya"
console.log("wo" + "ah"); // Prints "woah"
console.log("I love to " + "code."); // Prints "I love to code."
\end{lstlisting}
This process of appending one string to another is called \textit{concatenation}. Notice in the third example we had to make sure to include a space at the end of the first string. The computer will join the strings exactly, so we needed to make sure to include the space we wanted between the two strings.
\begin{lstlisting}
console.log("front " + "space"); // Prints "front space"
console.log("back" + " space"); // Prints "back space"
console.log("no" + "space"); // Prints "nospace"
console.log("middle" + " " + "space"); // Prints "middle space"
\end{lstlisting}
Just like with regular math, we can combine, or chain, our operations to get a final result:
\begin{lstlisting}
console.log("One" + ", " + "two" + ", " + "three!"); // Prints "One, two, three!"
\end{lstlisting}
\subsection{Properties}
When you introduce a new piece of data into a JavaScript program, the browser saves it as an instance of the data type. Every string instance has a property called \colorbox{lightgray}{length} that stores the number of characters in that string. You can retrieve property information by appending the string with a period and the property name:
\begin{lstlisting}
console.log("Hello".length); // Prints 5
\end{lstlisting}
The \colorbox{lightgray}{$.$} is another operator! We call it the dot operator. In the example above, the value saved to the \colorbox{lightgray}{length} property is retrieved from the instance of the string, \colorbox{lightgray}{'Hello'}. The program prints \colorbox{lightgray}{5} to the console, because \colorbox{lightgray}{Hello} has five characters in it.

\subsection{Methods}
Remember that methods are actions we can perform. JavaScript provides a number of string methods. We \textit{call}, or use, these methods by appending an instance with:
\begin{itemize}[leftmargin = *]
\item a period (the dot operator)
\item the name of the method
\item opening and closing parentheses
\end{itemize}
E.g. \colorbox{lightgray}{`example string'.methodName()}. \\
The syntax looks familiar. When we use \colorbox{lightgray}{console.log()} we’re calling the \colorbox{lightgray}{.log()} method on the \colorbox{lightgray}{console} object. Let us see \colorbox{lightgray}{console.log()} and some real string methods in action!
\begin{lstlisting}
console.log("hello".toUpperCase()); // Prints "HELLO"
console.log("Hey".startsWith("H")); // Prints true
\end{lstlisting}
Let’s look at each of the lines above: 
\begin{itemize}[leftmargin = *]
\item On the first line, the \colorbox{lightgray}{.toUpperCase()} method is called on the string instance \colorbox{lightgray}{`hello'}. The result is logged to the console. This method returns a string in all capital letters: \colorbox{lightgray}{`HELLO'}.
\item On the second line, the \colorbox{lightgray}{.startsWith()} method is called on the string instance \colorbox{lightgray}{`Hey'}. This method also accepts the character \colorbox{lightgray}{`H'} as an input, or argument, between the parentheses. Since the string \colorbox{lightgray}{`Hey'} does start with the letter \colorbox{lightgray}{`H'}, the method returns the boolean \colorbox{lightgray}{true}.
\end{itemize}
You can find a list of built-in string methods in the JavaScript documentation. \\
\underline{https://developer.mozilla.org/en-US/docs/Web/JavaScript/Reference/Global\_Objects/String}\\
Developers use documentation as a reference tool. It describes JavaScript’s keywords, methods, and syntax.
\subsection{Built-in Objects}
In addition to \colorbox{lightgray}{console}, there are other objects built into JavaScript. \\
\underline{https://developer.mozilla.org/en-US/docs/Web/JavaScript/Reference/Global\_Objects} \\
Down the line, you’ll build your own objects, but for now these “built-in” objects are full of useful functionality.
For example, if you wanted to perform more complex mathematical operations than arithmetic, JavaScript has the built-in \colorbox{lightgray}{Math} object. \\
\newline
The great thing about objects is that they have methods! Let’s call the \colorbox{lightgray}{.random()} method from the built-in \colorbox{lightgray}{Math} object:
\begin{lstlisting}
console.log(Math.random()); // Prints a random number between 0 and 1
\end{lstlisting}
In the example above, we called the \colorbox{lightgray}{.random()} method by appending the object name with the dot operator, the name of the method, and opening and closing parentheses. This method returns a random number between 0 and 1. \\
\newline
To generate a random number between 0 and 50, we could multiply this result by 50, like so:
\begin{lstlisting}
Math.random() * 50;
\end{lstlisting}
The example above will likely evaluate to a decimal. To ensure the answer is a whole number, we can take advantage of another useful \colorbox{lightgray}{Math} method called \colorbox{lightgray}{Math.floor()}. \\
\newline
\colorbox{lightgray}{Math.floor()} takes a decimal number, and rounds down to the nearest whole number. You can use \colorbox{lightgray}{Math.floor()} to round down a random number like this:
\begin{lstlisting}
Math.floor(Math.random() * 50);
\end{lstlisting}
In this case: 
\begin{enumerate}[leftmargin = *]
\item \colorbox{lightgray}{Math.random} generates a random number between 0 and 1.
\item We then multiply that number by \colorbox{lightgray}{50}, so now we have a number between 0 and 50.
\item Then, \colorbox{lightgray}{Math.floor()} rounds the number down to the nearest whole number.
\end{enumerate}
If you wanted to see the number printed to the terminal, you would still need to use a \colorbox{lightgray}{console.log()} statement:
\begin{lstlisting}
console.log(Math.floor(Math.random() * 50)); // Prints a random whole number between 0 and 50
\end{lstlisting}
To see all of the properties and methods on the Math object, take a look at the documentation here. \\
\underline{https://developer.mozilla.org/en-US/docs/Web/JavaScript/Reference/Global\_Objects/Math} \\
\underline{https://developer.mozilla.org/en-US/docs/Web/JavaScript/Reference/Global\_Objects/Number}
\subsection{Review}
Let us take one more glance at the concepts we just learned:
\begin{itemize}[leftmargin = *]
\item Data is printed, or logged, to the console, a panel that displays messages, with \colorbox{lightgray}{console.log()}.
\item We can write single-line comments with \colorbox{lightgray}{//} and multi-line comments between \colorbox{lightgray}{/*} and \colorbox{lightgray}{*/}.
\item There are 7 fundamental data types in JavaScript: strings, numbers, booleans, null, undefined, symbol, and object.
\item Numbers are any number without quotes: \colorbox{lightgray}{23.8879}
\item Strings are characters wrapped in single or double quotes: \colorbox{lightgray}{`Sample String'}
\item The built-in arithmetic operators include \colorbox{lightgray}{$+$}, \colorbox{lightgray}{$-$}, \colorbox{lightgray}{$*$}, \colorbox{lightgray}{$/$}, and \colorbox{lightgray}{$\%$}.
\item Objects, including instances of data types, can have properties, stored information. The properties are denoted with a \colorbox{lightgray}{.} after the name of the object, for example: \colorbox{lightgray}{`Hello'.length}.
\item Objects, including instances of data types, can have methods which perform actions. Methods are called by appending the object or instance with a period, the method name, and parentheses. For example: \colorbox{lightgray}{`hello'.toUpperCase()}.
\item We can access properties and methods by using the \colorbox{lightgray}{.}, dot operator.
\item Built-in objects, including \colorbox{lightgray}{Math}, are collections of methods and properties that JavaScript provides.
\end{itemize}

\newpage
\section{Variables}
\subsection{Variables}
In programming, a \textit{variable} is a container for a value. You can think of variables as little containers for information that live in a computer’s memory. Information stored in variables, such as a username, account number, or even personalized greeting can then be found in memory. \\
\newline
Variables also provide a way of labeling data with a descriptive name, so our programs can be understood more clearly by the reader and ourselves. \\
\newline
In short, variables label and store data in memory. There are only a few things you can do with variables: 
\begin{enumerate}[leftmargin = *]
\item Create a variable with a descriptive name.
\item Store or update information stored in a variable.
\item Reference or “get” information stored in a variable.
\end{enumerate}
It is important to distinguish that variables are not values; they contain values and represent them with a name. Later, we will cover how to use the \colorbox{lightgray}{var}, \colorbox{lightgray}{let}, and \colorbox{lightgray}{const} keywords to create variables.

\subsection{Create a Variable: var} 
There were a lot of changes introduced in the ES6 version of JavaScript in 2015. One of the biggest changes was two new keywords, \colorbox{lightgray}{let} and \colorbox{lightgray}{const}, to create, or declare, variables. Prior to the ES6, programmers could only use the \colorbox{lightgray}{var} keyword to declare variables.
\begin{lstlisting}
var myName = "Arya";
console.log(myName); // Output: Arya
\end{lstlisting}
Let’s consider the example above:
\begin{enumerate}[leftmargin = *]
\item \colorbox{lightgray}{var}, short for variable, is a JavaScript \textit{keyword} that creates, or \textit{declares}, a new variable.
\item \colorbox{lightgray}{myName} is the variable’s name. Capitalizing in this way is a standard convention in JavaScript called \textit{camel casing}. In camel casing you group words into one, the first word is lowercase, then every word that follows will have its first letter uppercased. (e.g. camelCaseEverything).
\item \colorbox{lightgray}{=} is the assignment operator. It assigns the value (\colorbox{lightgray}{`Arya'}) to the variable (\colorbox{lightgray}{myName}).
\item \colorbox{lightgray}{`Arya'} is the value assigned (\colorbox{lightgray}{=}) to the variable \colorbox{lightgray}{myName}. You can also say that the \colorbox{lightgray}{myName} variable is initialized with a value of \colorbox{lightgray}{`Arya'}.
\item After the variable is declared, the string value \colorbox{lightgray}{`Arya'} is printed to the console by referencing the variable name: \colorbox{lightgray}{console.log(myName)}.
\end{enumerate}
There are a few general rules for naming variables:
\begin{itemize}[leftmargin = *]
\item Variable names cannot start with numbers.
\item Variable names are case sensitive, so \colorbox{lightgray}{myName} and \colorbox{lightgray}{myname} would be different variables. It is bad practice to create two variables that have the same name using different cases.
\item Variable names cannot be the same as \textit{keywords}. For a comprehensive list of keywords check out MDN’s keyword documentation. \\
\underline{https://developer.mozilla.org/en-US/docs/Web/JavaScript/Reference/Statements/var}
\end{itemize}
Later, we will learn why ES6’s \colorbox{lightgray}{let} and \colorbox{lightgray}{const} are the preferred variable keywords by many programmers. Because there is still a ton of code written prior to ES6, it is helpful to be familiar with the pre-ES6 \colorbox{lightgray}{var} keyword. To learn more about \colorbox{lightgray}{var} and the quirks associated with it, check out the MDN var documentation. \\
\underline{https://developer.mozilla.org/en-US/docs/Web/JavaScript/Reference/Statements/var}

\subsection{Create a Variable: let}
As mentioned before, the \colorbox{lightgray}{let} keyword was introduced in ES6. The \colorbox{lightgray}{let} keyword signals that the variable can be reassigned a different value. Take a look at the example: 
\begin{lstlisting}
let meal = "Enchiladas";
console.log(meal); // Output: Enchiladas
meal = "Burrito";
console.log(meal); // Output: Burrito
\end{lstlisting}
Another concept that we should be aware of when using \colorbox{lightgray}{let} (and even \colorbox{lightgray}{var}) is that we can declare a variable without assigning the variable a value. In such a case, the variable will be automatically initialized with a value of \colorbox{lightgray}{undefined}:
\begin{lstlisting}
let price;
console.log(price); // Output: undefined
price = 350;
console.log(price); // Output: 350
\end{lstlisting}
Notice in the example above: 
\begin{itemize}[leftmargin = *]
\item If we don’t assign a value to a variable declared using the \colorbox{lightgray}{let} keyword, it automatically has a value of \colorbox{lightgray}{undefined}.
\item We can reassign the value of the variable.
\end{itemize}

\subsection{Create a Variable: const}
The \colorbox{lightgray}{const} keyword was also introduced in ES6, and is short for the word constant. Just like with \colorbox{lightgray}{var} and \colorbox{lightgray}{let} you can store any value in a \colorbox{lightgray}{const} variable. The way you declare a \colorbox{lightgray}{const} variable and assign a value to it follows the same structure as \colorbox{lightgray}{let} and \colorbox{lightgray}{var}. Take a look at the following example: 
\begin{lstlisting}
const myName = "Gilberto";
console.log(myName); // Output: Gilberto
\end{lstlisting}
However, a \colorbox{lightgray}{const} variable cannot be reassigned because it is constant. If you try to reassign a \colorbox{lightgray}{const} variable, you’ll get a \colorbox{lightgray}{TypeError}. \\
\newline
Constant variables must be assigned a value when declared. If you try to declare a \colorbox{lightgray}{const} variable without a value, you’ll get a \colorbox{lightgray}{SyntaxError}. \\
\newline
If you’re trying to decide between which keyword to use, \colorbox{lightgray}{let} or \colorbox{lightgray}{const}, think about whether you’ll need to reassign the variable later on. If you do need to reassign the variable use \colorbox{lightgray}{let}, otherwise, use \colorbox{lightgray}{const}.

\subsection{Mathematical Assignment Operators}
Let us consider how we can use variables and math operators to calculate new values and assign them to a variable. Check out the example below:
\begin{lstlisting}
let w = 4;
w = w + 1;
console.log(w); // Output: 5
\end{lstlisting}
In the example above, we created the variable \colorbox{lightgray}{w} with the number \colorbox{lightgray}{4} assigned to it. The following line, \colorbox{lightgray}{w = w + 1}, increases the value of \colorbox{lightgray}{w} from \colorbox{lightgray}{4} to \colorbox{lightgray}{5}. \\
\newline
Another way we could have reassigned \colorbox{lightgray}{w} after performing some mathematical operation on it is to use built-in \textit{mathematical assignment operators}. We could re-write the code above to be:
\begin{lstlisting}
let w = 4;
w += 1;
console.log(w); // Output: 5
\end{lstlisting}
In the second example, we used the \colorbox{lightgray}{+=} assignment operator to reassign \colorbox{lightgray}{w}. We’re performing the mathematical operation of the first operator \colorbox{lightgray}{+} using the number to the right, then reassigning \colorbox{lightgray}{w} to the computed value. \\
\newline
We also have access to other mathematical assignment operators: \colorbox{lightgray}{-=}, \colorbox{lightgray}{*=}, and \colorbox{lightgray}{/=} which work in a similar fashion. 
\begin{lstlisting}
let x = 20;
x -= 5; // Can be written as x = x - 5
console.log(x); // Output: 15

let y = 50;
y *= 2; // Can be written as y = y * 2
console.log(y); // Output: 100

let z = 8;
z /= 2; // Can be written as z = z / 2
console.log(z); // Output: 4 
\end{lstlisting}

\subsection{The Increment and Decrement Operator}
Other mathematical assignment operators include the increment operator (\colorbox{lightgray}{++}) and decrement operator (\colorbox{lightgray}{$--$}). The increment operator will increase the value of the variable by 1. The decrement operator will decrease the value of the variable by 1. For example:
\begin{lstlisting}
let a = 10;
a++;
console.log(a); // Output: 11
\end{lstlisting}
\begin{lstlisting}
let b = 20;
b--;
console.log(b); // Output: 19
\end{lstlisting}
Just like the previous mathematical assignment operators (\colorbox{lightgray}{+=}, \colorbox{lightgray}{$-$=}, \colorbox{lightgray}{*=}, \colorbox{lightgray}{/=}), the variable’s value is updated and assigned as the new value of that variable.

\subsection{String Concatenation with Variables}
Before we assigned strings to variables. Now, let us go over how to connect, or concatenate, strings in variables. The \colorbox{lightgray}{+} operator can be used to combine two string values even if those values are being stored in variables:
\begin{lstlisting}
let myPet = "armadillo";
console.log("I own a pet " + myPet + "."); // Output: "I own a pet armadillo."
\end{lstlisting}
In the example above, we assigned the value \colorbox{lightgray}{`armadillo'} to the \colorbox{lightgray}{myPet} variable. On the second line, the \colorbox{lightgray}{+} operator is used to combine three strings: \colorbox{lightgray}{`I own a pet'}, the value saved to \colorbox{lightgray}{myPet}, and \colorbox{lightgray}{`.'}. We log the result of this concatenation to the console as:
\begin{lstlisting}
I own a pet armadillo.
\end{lstlisting}

\subsection{String Interpolation}
In the ES6 version of JavaScript, we can insert, or \textit{interpolate}, variables into strings using \textit{template literals}. Check out the following example where a template literal is used to log strings together:
\textasciigrave
\begin{lstlisting}
const myPet = "armadillo";
console.log((*@\texttt{\textasciigrave}@*)I own a pet ${myPet}.(*@\texttt{\textasciigrave}@*)); // Output: I own a pet armadillo.
\end{lstlisting}
Notice that:
\begin{itemize}[leftmargin = *]
\item a template literal is wrapped by backticks \colorbox{lightgray}{\textasciigrave} (this key is usually located on the top of your keyboard, left of the \fbox{1} key).
\item Inside the template literal, you’ll see a placeholder, \colorbox{lightgray}{\$\{myPet\}}. The value of \colorbox{lightgray}{myPet} is inserted into the template literal.
\item When we interpolate \colorbox{lightgray}{\textasciigrave I own a pet \$\{myPet\}.\textasciigrave}, the output we print is the string: \colorbox{lightgray}{`I own a pet armadillo.'}
\end{itemize}
One of the biggest benefits to using template literals is the readability of the code. Using template literals, you can more easily tell what the new string will be. You also do not have to worry about escaping double quotes or single quotes. 

\subsection{typeof operator}
While writing code, it can be useful to keep track of the data types of the variables in your program. If you need to check the data type of a variable’s value, you can use the \colorbox{lightgray}{typeof} operator. The \colorbox{lightgray}{typeof} operator checks the value to its right and \textit{returns}, or passes back, a string of the data type. 
\begin{lstlisting}
const unknown1 = "foo";
console.log(typeof unknown1); // Output: string

const unknown2 = 10;
console.log(typeof unknown2); // Output: number

const unknown3 = true; 
console.log(typeof unknown3); // Output: boolean
\end{lstlisting}
Let us break down the first example. Since the value \colorbox{lightgray}{unknown1} is \colorbox{lightgray}{`foo'}, a string, \colorbox{lightgray}{typeof unknown1} will return \colorbox{lightgray}{`string'}.

\subsection{Review Variables}
Variables is a powerful concept you will use in all your future programming endeavors. Let us take one more glance at the concepts we just learned:
\begin{itemize}[leftmargin = *]
\item Variables hold reusable data in a program and associate it with a name. 
\item Variables are stored in memory.
\item The \colorbox{lightgray}{var} keyword is used in pre-ES6 versions of JS. 
\item \colorbox{lightgray}{let} is the preferred way to declare a variable when it can be reassigned, and \colorbox{lightgray}{const} is the preferred way to declare a variable with a constant value.
\item Variables that have not been initialized store the primitive data type \colorbox{lightgray}{undefined}. 
\item Mathematical assignment operators make it easy to calculate a new value and assign it to the same variable. 
\item The \colorbox{lightgray}{+} operator is used to concatenate strings including string values held in variables. 
\item In ES6, template literals use backticks \colorbox{lightgray}{\textasciigrave} and \colorbox{lightgray}{\$\{\}} to interpolate values into a string.
\item The \colorbox{lightgray}{typeof} keyword returns the data type (as a string) of a value.
\end{itemize}

\newpage
\section{Conditional Statements}
\subsection{What are Conditional Statements?}
In life, we make decisions based on circumstances. Think of an everyday decision as mundane as falling asleep — if we are tired, we go to bed, otherwise, we wake up and start our day. 
These if-else decisions can be modeled in code by creating \textit{conditional statements}. A conditional statement checks a specific condition(s) and performs a task based on the condition(s). \\
\newline
Now we will explore how programs make decisions by evaluating conditions and introduce logic into our code. We will cover the following concepts:
\begin{itemize}[leftmargin = *]
\item \colorbox{lightgray}{if}, \colorbox{lightgray}{else if}, and \colorbox{lightgray}{else} statements 
\item comparison operators
\item logical operators
\item truthy vs falsy values
\item ternary operators
\item \colorbox{lightgray}{switch} statement
\end{itemize}

\subsection{If Statement}
We often perform a task based on a condition. For example, if the weather is nice today, then we will go outside. If the alarm clock rings, then we will shut it off. If we are tired, then we will go to sleep. \\
\newline
In programming, we can also perform a task based on a condition using an \colorbox{lightgray}{if} statement:
\begin{lstlisting}
if (true) {
  console.log("This message will print!"); 
}
// Prints: This message will print!
\end{lstlisting}
Notice in the example above, we have an \colorbox{lightgray}{if} statement. The \colorbox{lightgray}{if} statement is composed of:
\begin{itemize}[leftmargin = *]
\item The \colorbox{lightgray}{if} keyword followed by a set of parentheses \colorbox{lightgray}{()} which is followed by a \textit{code block}, or \textit{block statement}, indicated by a set of curly braces \colorbox{lightgray}{\{\}}.
\item Inside the parentheses \colorbox{lightgray}{()}, a condition is provided that evaluates to \colorbox{lightgray}{true} or \colorbox{lightgray}{false}.
\item If the condition evaluates to \colorbox{lightgray}{true}, the code inside the curly braces \colorbox{lightgray}{\{\}} runs, or \textit{executes}.
\item If the condition evaluates to \colorbox{lightgray}{false}, the block will not execute.
\end{itemize}

\subsection{If...Else Statements}
In the previous chapter, we used an \colorbox{lightgray}{if} statement that checked a condition to decide whether or not to run a block of code. In many cases, we’ll have code we want to run if our condition evaluates to \colorbox{lightgray}{false}. \\
\newline
If we wanted to add some default behavior to the \colorbox{lightgray}{if} statement, we can add an \colorbox{lightgray}{else} statement to run a block of code when the condition evaluates to \colorbox{lightgray}{false}. Take a look at the inclusion of an \colorbox{lightgray}{else} statement: 
\begin{lstlisting}
if (false) {
  console.log("The code in this block will not run.");
} else {
  console.log("But the code in this block will!");
}

// Prints: But the code in this block will!
\end{lstlisting}
An \colorbox{lightgray}{else} statement must be paired with an \colorbox{lightgray}{if} statement, and together they are referred to as an \colorbox{lightgray}{if...else} statement. In the example above, the \colorbox{lightgray}{else} statement:
\begin{itemize}[leftmargin = *]
\item Uses the \colorbox{lightgray}{else} keyword following the code block of an \colorbox{lightgray}{if} statement.
\item Has a code block that is wrapped by a set of curly braces \colorbox{lightgray}{\{\}}.
\item The code inside the \colorbox{lightgray}{else} statement code block will execute when the \colorbox{lightgray}{if} statement’s condition evaluates to false.
\end{itemize}
\colorbox{lightgray}{if...else} statements allow us to automate solutions to yes-or-no questions, also known as binary decisions.

\subsection{Comparison Operators}
When writing conditional statements, sometimes we need to use different types of operators to compare values. These operators are called \textit{comparison operators}. Here is a list of some handy comparison operators and their syntax:
\begin{itemize}[leftmargin = *]
\item Less than: \colorbox{lightgray}{$<$}
\item Greater than: \colorbox{lightgray}{$>$}
\item Less than or equal to: \colorbox{lightgray}{$<$$=$}
\item Greater than or equal to: \colorbox{lightgray}{$>$$=$}
\item Is equal to: \colorbox{lightgray}{$=$$=$$=$}
\item Is not equal to: \colorbox{lightgray}{$!$$=$$=$}
\end{itemize}
Comparison operators compare the value on the left with the value on the right. For instance:
\begin{lstlisting}
10 < 12 // Evaluates to true
\end{lstlisting}
It can be helpful to think of comparison statements as questions. When the answer is “yes”, the statement evaluates to \colorbox{lightgray}{true}, and when the answer is “no”, the statement evaluates to \colorbox{lightgray}{false}. The code above would be asking: is 10 less than 12? Yes! So \colorbox{lightgray}{10 < 12} evaluates to \colorbox{lightgray}{true}. \\
\newline
We can also use comparison operators on different data types like strings:
\begin{lstlisting}
"apples" === "oranges" // false
\end{lstlisting}
In the example above, we’re using the identity operator (\colorbox{lightgray}{$=$$=$$=$}) to check if the string \colorbox{lightgray}{`apples'} is the same as the string \colorbox{lightgray}{`oranges'}. Since the two strings are not the same, the comparison statement evaluates to \colorbox{lightgray}{false}. \\
\newline
All comparison statements evaluate to either true or false and are made up of: 
\begin{itemize}[leftmargin = *]
\item Two values that will be compared.
\item An operator that separates the values and compares them accordingly (\colorbox{lightgray}{$>$}, \colorbox{lightgray}{$<$}, \colorbox{lightgray}{$<$$=$}, \colorbox{lightgray}{$>$$=$}, \colorbox{lightgray}{$=$$=$$=$} ,\colorbox{lightgray}{$!$$=$$=$}).
\end{itemize}
The difference between \colorbox{lightgray}{$=$$=$} and \colorbox{lightgray}{$=$$=$$=$} is that \colorbox{lightgray}{$=$$=$$=$}  is known as Identity / strict equality, meaning that the data types also must be equal. 
\begin{lstlisting}
console.log(1 == true) // output: true
console.log(1 === true) // output: false
\end{lstlisting}

\subsection{Logical Operators}
Working with conditionals means that we will be using booleans, \colorbox{lightgray}{true} or \colorbox{lightgray}{false} values. In JavaScript, there are operators that work with boolean values known as \textit{logical operators}. We can use logical operators to add more sophisticated logic to our conditionals. There are three logical operators: 
\begin{itemize}[leftmargin = *]
\item the and operator (\colorbox{lightgray}{$\&\&$})
\item the or operator (\colorbox{lightgray}{||})
\item the not operator, otherwise known as the bang operator (\colorbox{lightgray}{!}) 
\end{itemize}
When we use the \colorbox{lightgray}{$\&\&$} operator, we are checking that two things are true:
\begin{lstlisting}
if (stopLight === "green" && pedestrians === 0) {
  console.log("Go!");
} else {
  console.log("Stop");
}
\end{lstlisting}
When using the \colorbox{lightgray}{\&\&} operator, both conditions must evaluate to \colorbox{lightgray}{true} for the entire condition to evaluate to \colorbox{lightgray}{true} and execute. Otherwise, if either condition is \colorbox{lightgray}{false}, the \colorbox{lightgray}{\&\&} condition will evaluate to \colorbox{lightgray}{false} and the \colorbox{lightgray}{else} block will execute. \\
\newline
If we only care about either condition being true, we can use the \colorbox{lightgray}{||} operator:
\begin{lstlisting}
if (day === "Saturday" || day === "Sunday") {
  console.log("Enjoy the weekend!");
} else {
  console.log("Do some work.");
}
\end{lstlisting}
When using the \colorbox{lightgray}{||} operator, only one of the conditions must evaluate to \colorbox{lightgray}{true} for the overall statement to evaluate to \colorbox{lightgray}{true}. In the code example above, if either \colorbox{lightgray}{day $=$$=$$=$ `Saturday'} or \colorbox{lightgray}{day $=$$=$$=$ `Sunday'} evaluates to \colorbox{lightgray}{true} the \colorbox{lightgray}{if}‘s condition will evaluate to \colorbox{lightgray}{true} and its code block will execute. If the first condition in an \colorbox{lightgray}{||} statement evaluates to \colorbox{lightgray}{true}, the second condition will not even be checked. Only if \colorbox{lightgray}{day $=$$=$$=$ 'Saturday'} evaluates to \colorbox{lightgray}{false} will \colorbox{lightgray}{day $=$$=$$=$ 'Sunday'} be evaluated. The code in the \colorbox{lightgray}{else} statement above will execute only if both comparisons evaluate to \colorbox{lightgray}{false}. \\
\newline
The \colorbox{lightgray}{!} \textit{not operator reverses}, or \textit{negates}, the value of a boolean:
\begin{lstlisting}
let excited = true;
console.log(!excited); // Prints false

let sleepy = false;
console.log(!sleepy); // Prints true
\end{lstlisting}
Essentially, the \colorbox{lightgray}{!} operator will either take a \colorbox{lightgray}{true} value and pass back \colorbox{lightgray}{false}, or it will take a \colorbox{lightgray}{false} value and pass back \colorbox{lightgray}{true}. \\
\newline
Logical operators are often used in conditional statements to add another layer of logic to our code.

\subsection{Truthy and Falsy}
Let us consider how non-boolean data types, like strings or numbers, are evaluated when checked inside a condition. \\
\newline
Sometimes, you will want to check if a variable exists and you will not necessarily want it to equal a specific value — you’ll only check to see if the variable has been assigned a value. Here is an example:
\begin{lstlisting}
let myVariable = "I Exist!";

if (myVariable) {
   console.log(myVariable)
} else {
   console.log("The variable does not exist.")
}
\end{lstlisting}
The code block in the \colorbox{lightgray}{if} statement will run because \colorbox{lightgray}{myVariable} has a \textit{truthy} value; even though the value of \colorbox{lightgray}{myVariable} is not explicitly the value \colorbox{lightgray}{true}, when used in a boolean or conditional context, it evaluates to \colorbox{lightgray}{true} because it has been assigned a non-falsy value. \\
\newline
So which values are \textit{falsy}— or evaluate to \colorbox{lightgray}{false} when checked as a condition? The list of falsy values includes:
\begin{itemize}[leftmargin = *]
\item \colorbox{lightgray}{0}
\item Empty strings like \colorbox{lightgray}{``''} or \colorbox{lightgray}{`'}
\item \colorbox{lightgray}{null} which represent when there is no value at all
\item \colorbox{lightgray}{undefined} which represent when a declared variable lacks a value
\item \colorbox{lightgray}{NaN}, or Not a Number
\end{itemize}
Here’s an example with numbers: 
\begin{lstlisting}
let numberOfApples = 0;

if (numberOfApples){
   console.log("Let us eat apples!");
} else {
   console.log("No apples left!");
}

// Prints 'No apples left!'
\end{lstlisting}
The condition evaluates to \colorbox{lightgray}{false} because the value of the \colorbox{lightgray}{numberOfApples} is \colorbox{lightgray}{0}. Since \colorbox{lightgray}{0} is a falsy value, the code block in the \colorbox{lightgray}{else} statement will run.

\subsection{Truthy and Falsy Assignment}
Truthy and falsy evaluations open a world of short-hand possibilities. Say you have a website and want to take a user’s username to make a personalized greeting. Sometimes, the user does not have an account, making the \colorbox{lightgray}{username} variable falsy. The code below checks if \colorbox{lightgray}{username} is defined and assigns a default string if it is not:
\begin{lstlisting}
let defaultName;
if (username) {
  defaultName = username;
} else {
  defaultName = "Stranger";
}
\end{lstlisting}
If you combine your knowledge of logical operators you can use a short-hand for the code above. In a boolean condition, JavaScript assigns the truthy value to a variable if you use the \colorbox{lightgray}{||} operator in your assignment:
\begin{lstlisting}
let defaultName = username || "Stranger";
\end{lstlisting}
Because \colorbox{lightgray}{||} or statements check the left-hand condition first, the variable \colorbox{lightgray}{defaultName} will be assigned the actual value of \colorbox{lightgray}{username} if is truthy, and it will be assigned the value of \colorbox{lightgray}{`Stranger'} if \colorbox{lightgray}{username} is falsy. This concept is also referred to as \textit{short-circuit evaluation}.

\subsection{Ternary Operator}
In the spirit of using short-hand syntax, we can use a \textit{ternary operator} to simplify an \colorbox{lightgray}{if...else} statement. Take a look at the \colorbox{lightgray}{if...else} statement example: 
\begin{lstlisting}
let isNightTime = true;

if (isNightTime) {
  console.log("Turn on the lights!");
} else {
  console.log("Turn off the lights!");
}
\end{lstlisting}
We can use a \textit{ternary operator} to perform the same functionality:
\begin{lstlisting}
isNightTime ? console.log("Turn on the lights!") : console.log("Turn off the lights!");
\end{lstlisting}
In the example above:
\begin{itemize}[leftmargin = *]
\item The condition, \colorbox{lightgray}{isNightTime}, is provided before the \colorbox{lightgray}{?}.
\item Two expressions follow the \colorbox{lightgray}{?} and are separated by a colon \colorbox{lightgray}{:}.
\item If the condition evaluates to \colorbox{lightgray}{true}, the first expression executes.
\item If the condition evaluates to \colorbox{lightgray}{false}, the second expression executes.
\end{itemize}
Like \colorbox{lightgray}{if...else} statements, ternary operators can be used for conditions which evaluate to \colorbox{lightgray}{true} or \colorbox{lightgray}{false}.

\subsection{Else If Statements}
We can add more conditions to our \colorbox{lightgray}{if...else} with an \colorbox{lightgray}{else if} statement. The \colorbox{lightgray}{else if} statement allows for more than two possible outcomes. You can add as many \colorbox{lightgray}{else if} statements as you would like, to make more complex conditionals! \\
\newline
The \colorbox{lightgray}{else if} statement always comes after the \colorbox{lightgray}{if} statement and before the \colorbox{lightgray}{else} statement. The \colorbox{lightgray}{else if} statement also takes a condition. Let us take a look at the syntax:
\begin{lstlisting}
let stopLight = "yellow";

if (stopLight === "red") {
  console.log("Stop!");
} else if (stopLight === "yellow") {
  console.log("Slow down.");
} else if (stopLight === "green") {
  console.log("Go!");
} else {
  console.log("Caution, unknown!");
}
\end{lstlisting}
The \colorbox{lightgray}{else if} statements allow you to have multiple possible outcomes. \colorbox{lightgray}{if/else if/else} statements are read from top to bottom, so the first condition that evaluates to \colorbox{lightgray}{true} from the top to bottom is the block that gets executed. \\
\newline
In the example above, since \colorbox{lightgray}{stopLight $=$$=$$=$ `red'} evaluates to false and \colorbox{lightgray}{stopLight $=$$=$$=$ `yellow'} evaluates to \colorbox{lightgray}{true}, the code inside the first \colorbox{lightgray}{else if} statement is executed. The rest of the conditions are not evaluated. If none of the conditions evaluated to true, then the code in the \colorbox{lightgray}{else} statement would have executed.

\subsection{The switch keyword}
\colorbox{lightgray}{else if} statements are a great tool if we need to check multiple conditions. In programming, we often find ourselves needing to check multiple values and handling each of them differently. For example:
\begin{lstlisting}
let groceryItem = "papaya";

if (groceryItem === "tomato") {
  console.log("Tomatoes are $0.49");
} else if (groceryItem === "papaya"){
  console.log("Papayas are $1.29");
} else {
  console.log("Invalid item");
}
\end{lstlisting}
In the code above, we have a series of conditions checking for a value that matches a \colorbox{lightgray}{groceryItem} variable. Our code works fine, but imagine if we needed to check 100 different values. Having to write that many \colorbox{lightgray}{else if} statements sounds like a pain. \\
\newline
A \colorbox{lightgray}{switch} statement provides an alternative syntax that is easier to read and write. A \colorbox{lightgray}{switch} statement looks like this:
\begin{lstlisting}
let groceryItem = "papaya";
switch (groceryItem) {
  case "tomato":
    console.log("Tomatoes are $0.49");
    break;
  case "lime":
    console.log("Limes are $1.49");
    break;
  case "papaya":
    console.log("Papayas are $1.29");
    break;
  default:
    console.log("Invalid item");
    break;
}
// Prints "Papayas are $1.29"
\end{lstlisting}
\begin{itemize}[leftmargin = *]
\item The \colorbox{lightgray}{switch} keyword initiates the statement and is followed by \colorbox{lightgray}{( ... )}, which contains the value that each \colorbox{lightgray}{case} will compare. In the example, the value or expression of the \colorbox{lightgray}{switch} statement is \colorbox{lightgray}{groceryItem}.
\item Inside the block, \colorbox{lightgray}{\{ ... \}}, there are multiple \colorbox{lightgray}{case}s. The \colorbox{lightgray}{case} keyword checks if the expression matches the specified value that comes after it. The value following the first \colorbox{lightgray}{case} is \colorbox{lightgray}{`tomato'}. If the value of \colorbox{lightgray}{groceryItem} equalled \colorbox{lightgray}{`tomato'}, that \colorbox{lightgray}{case}'s \colorbox{lightgray}{console.log()} would run.
\item The value of \colorbox{lightgray}{groceryItem} is \colorbox{lightgray}{`papaya'}, so the third \colorbox{lightgray}{case} runs— \colorbox{lightgray}{Papayas are \$1.29} is logged to the console.
\item The \colorbox{lightgray}{break} keyword tells the computer to exit the block and not execute any more code or check any other cases inside the code block. Note: Without the \colorbox{lightgray}{break} keyword at the end of each case, the program would execute the code for all matching cases and the default code as well. This behavior is different from \colorbox{lightgray}{if}/\colorbox{lightgray}{else} conditional statements which execute only one block of code.
\item At the end of each \colorbox{lightgray}{switch} statement, there is a \colorbox{lightgray}{default} statement. If none of the \colorbox{lightgray}{case}s are true, then the code in the \colorbox{lightgray}{default} statement will run.
\end{itemize}

\subsection{Review}
Here are some of the major concepts for conditionals:
\begin{itemize}[leftmargin = *]
\item An \colorbox{lightgray}{if} statement checks a condition and will execute a task if that condition evaluates to \colorbox{lightgray}{true}.
\item \colorbox{lightgray}{if...else} statements make binary decisions and execute different code blocks based on a provided condition.
\item We can add more conditions using \colorbox{lightgray}{else if} statements.
\item Comparison operators, including \colorbox{lightgray}{$<$}, \colorbox{lightgray}{$>$}, \colorbox{lightgray}{$<$$=$}, \colorbox{lightgray}{$>$$=$}, \colorbox{lightgray}{$=$$=$$=$}, and \colorbox{lightgray}{$!$$=$$=$} can compare two values.
\item The logical and operator, \colorbox{lightgray}{\&\&}, or “and”, checks if both provided expressions are truthy.
\item The logical operator \colorbox{lightgray}{||}, or “or”, checks if either provided expression is truthy.
\item The bang operator, \colorbox{lightgray}{!}, switches the truthiness and falsiness of a value.
\item The ternary operator is shorthand to simplify concise \colorbox{lightgray}{if...else} statements.
\item A \colorbox{lightgray}{switch} statement can be used to simplify the process of writing multiple \colorbox{lightgray}{else if} statements. The \colorbox{lightgray}{break} keyword stops the remaining \colorbox{lightgray}{case}s from being checked and executed in a \colorbox{lightgray}{switch} statement.
\end{itemize}

\newpage
\section{Functions}
\subsection{What are Functions?}
When first learning how to calculate the area of a rectangle, there is a sequence of steps to calculate the correct answer:
\begin{enumerate}[leftmargin = *]
\item Measure the width of the rectangle.
\item Measure the height of the rectangle.
\item Multiply the width and height of the rectangle.
\end{enumerate}
With practice, you can calculate the area of the rectangle without being instructed with these three steps every time. We can calculate the area of one rectangle with the following code: 
\begin{lstlisting}
const width = 10;
const height = 6;
const area =  width * height;
console.log(area); // Output: 60
\end{lstlisting}
Imagine being asked to calculate the area of three different rectangles: 
\begin{lstlisting}
// Area of the first rectangle
const width1 = 10;
const height1 = 6;
const area1 =  width1 * height1;

// Area of the second rectangle
const width2 = 4;
const height2 = 9;
const area2 =  width2 * height2;

// Area of the third rectangle
const width3 = 10;
const height3 = 10;
const area3 =  width3 * height3;
\end{lstlisting}
In programming, we often use code to perform a specific task multiple times. Instead of rewriting the same code, we can group a block of code together and associate it with one task, then we can reuse that block of code whenever we need to perform the task again. We achieve this by creating a \textit{function}. A function is a reusable block of code that groups together a sequence of statements to perform a specific task. \\
\newline
Now, you will learn how to create and use functions, and how they can be used to create clearer and more concise code.

\newpage
\subsection{Function Declarations} 
In JavaScript, there are many ways to create a function. One way to create a function is by using a \textit{function declaration}. Just like how a variable declaration binds a value to a variable name, a function declaration binds a function to a name, or an \textit{identifier}. Take a look at the anatomy of a function declaration below:
\vspace{-4mm}
\begin{figure}[H]
\includegraphics[scale = 0.8]{4_1}
\centering
\end{figure}
\vspace{-4mm}
A function declaration consists of: 
\begin{itemize}[leftmargin = *]
\item The \colorbox{lightgray}{function} keyword.
\item The name of the function, or its identifier, followed by parentheses.
\item A function body, or the block of statements required to perform a specific task, enclosed in the function’s curly brackets, \colorbox{lightgray}{\{ \}}.
\end{itemize}
A function declaration is a function that is bound to an identifier, or name. In the next chapter, we will go over how to run the code inside the function body.
We should also be aware of the \textit{hoisting} feature in JavaScript which allows access to function declarations before they’re defined. Take a look at example of hoisting:
\begin{lstlisting}
console.log(greetWorld()); // Output: Hello, World!

function greetWorld() {
  console.log("Hello, World!");
}
\end{lstlisting}
Notice how hoisting allowed \colorbox{lightgray}{greetWorld()} to be called before the \colorbox{lightgray}{greetWorld()} function was defined! Since hoisting is not considered good practice, we simply want you to be aware of this feature.

\newpage
\subsection{Calling a Function}
As we saw in the previous chapter, a function declaration binds a function to an identifier. However, a function declaration does not ask the code inside the function body to run, it just declares the existence of the function. The code inside a function body runs, or \textit{executes}, only when the function is \textit{called}. To call a function in your code, you type the function name followed by parentheses.
\vspace{-2mm}
\begin{figure}[H]
\includegraphics[scale = 0.65]{4_2}
\centering
\end{figure}
\vspace{-3mm}
This \textit{function call} executes the function body, or all of the statements between the curly braces in the function declaration.
\begin{figure}[H]
\includegraphics[scale = 0.6]{4_3}
\centering
\end{figure}
We can call the same function as many times as needed.

\subsection{Parameters and Arguments}
So far, the functions we have created execute a task without an input. However, some functions can take inputs and use the inputs to perform a task. When declaring a function, we can specify its \textit{parameters}. Parameters allow functions to accept input(s) and perform a task using the input(s). We use parameters as placeholders for information that will be passed to the function when it is called. \\
\newline
Let’s observe how to specify parameters in our function declaration:
\begin{figure}[H]
\includegraphics[scale = 0.75]{4_4}
\centering
\end{figure}
In the diagram above, \colorbox{lightgray}{calculateArea()}, computes the area of a rectangle, based on two inputs, \colorbox{lightgray}{width} and \colorbox{lightgray}{height}. The parameters are specified between the parenthesis as \colorbox{lightgray}{width} and \colorbox{lightgray}{height}, and inside the function body, they act just like regular variables. \colorbox{lightgray}{width} and \colorbox{lightgray}{height} act as placeholders for values that will be multiplied together. \\
\newline
When calling a function that has parameters, we specify the values in the parentheses that follow the function name. The values that are passed to the function when it is called are called \textit{arguments}. Arguments can be passed to the function as values or variables.
\begin{figure}[H]
\includegraphics[scale = 0.65]{4_5}
\centering
\end{figure}
In the function call above, the number \colorbox{lightgray}{10} is passed as the \colorbox{lightgray}{width} and \colorbox{lightgray}{6} is passed as \colorbox{lightgray}{height}. Notice that the order in which arguments are passed and assigned follows the order that the parameters are declared.
\begin{figure}[H]
\includegraphics[scale = 0.73]{4_6}
\centering
\end{figure}
The variables \colorbox{lightgray}{rectWidth} and \colorbox{lightgray}{rectHeight} are initialized with the values for the height and width of a rectangle before being used in the function call. \\
\newline
By using parameters, \colorbox{lightgray}{calculateArea()} can be reused to compute the area of any rectangle! Functions are a powerful tool in computer programming so let’s practice creating and calling functions with parameters.

\subsection{Default Parameters}
One of the features added in ES6 is the ability to use \textit{default parameters}. Default parameters allow parameters to have a predetermined value in case there is no argument passed into the function or if the argument is \colorbox{lightgray}{undefined} when called. \\
\newline
Take a look at the code snippet below that uses a default parameter:
\begin{lstlisting}
function greeting (name = "stranger") {
  console.log((*@\texttt{\textasciigrave}@*)Hello, ${name}!(*@\texttt{\textasciigrave}@*))
}

greeting("Nick") // Output: Hello, Nick!
greeting() // Output: Hello, stranger!
\end{lstlisting}
\begin{itemize}[leftmargin = *]
\item In the example above, we used the \colorbox{lightgray}{=} operator to assign the parameter \colorbox{lightgray}{name} a default value of \colorbox{lightgray}{`stranger'}. This is useful to have in case we ever want to include a non-personalized default greeting!
\item When the code calls \colorbox{lightgray}{greeting(`Nick')} the value of the argument is passed in and, \colorbox{lightgray}{`Nick'}, will override the default parameter of \colorbox{lightgray}{`stranger'} to log \colorbox{lightgray}{`Hello, Nick!'} to the console.
\item When there is not an argument passed into \colorbox{lightgray}{greeting()}, the default value of \colorbox{lightgray}{`stranger'} is used, and \colorbox{lightgray}{`Hello, stranger!'} is logged to the console.
\end{itemize}
By using a default parameter, we account for situations when an argument isn’t passed into a function that is expecting an argument.

\subsection{Return}
When a function is called, the computer will run through the function’s code and evaluate the result of calling the function. By default that resulting value is \colorbox{lightgray}{undefined}. 
\begin{lstlisting}
function rectangleArea(width, height) {
  let area = width * height;
}
console.log(rectangleArea(5, 7)) // Prints undefined
\end{lstlisting}
In the code example, we defined our function to calculate the \colorbox{lightgray}{area} of a \colorbox{lightgray}{width} and \colorbox{lightgray}{height} parameter. Then \colorbox{lightgray}{rectangleArea()} is invoked with the arguments \colorbox{lightgray}{5} and \colorbox{lightgray}{7}. But when we went to print the results we got \colorbox{lightgray}{undefined}. The computer did calculate the area as \colorbox{lightgray}{35}, but we did not capture it. But we can capture with the keyword \colorbox{lightgray}{return}. 
\begin{figure}[H]
\includegraphics[scale = 0.73]{4_7}
\centering
\end{figure}
To pass back information from the function call, we use a return statement. To create a return statement, we use the \colorbox{lightgray}{return} keyword followed by the value that we wish to return. Like we saw above, if the value is omitted, \colorbox{lightgray}{undefined} is returned instead. \\
\newline
When a \colorbox{lightgray}{return} statement is used in a function body, the execution of the function is stopped and the code that follows it will not be executed. Look at the example below:
\begin{lstlisting}
function rectangleArea(width, height) {
  if (width < 0 || height < 0) {
    return "You need positive integers to calculate area!";
  }
  return width * height;
}
\end{lstlisting}
If an argument for \colorbox{lightgray}{width} or \colorbox{lightgray}{height} is less than \colorbox{lightgray}{0}, then \colorbox{lightgray}{rectangleArea()} will return the string \\ 
\colorbox{lightgray}{`You need positive integers to calculate area!'}. The second return statement \colorbox{lightgray}{width * height} will not run.

The \colorbox{lightgray}{return} keyword is powerful because it allows functions to produce an output. We can then save the output to a variable for later use.

\subsection{Helper Functions}
We can also use the return value of a function inside another function. These functions being called within another function are often referred to as \textit{helper functions}. Since each function is carrying out a specific task, it makes our code easier to read and debug if necessary. \\
\newline
If we wanted to define a function that converts the temperature from Celsius to Fahrenheit, we could write two functions like:
\begin{lstlisting}
function multiplyByNineFifths(number) {
  return number * (9/5);
};

function getFahrenheit(celsius) {
  return multiplyByNineFifths(celsius) + 32;
};

getFahrenheit(15); // Returns 59
\end{lstlisting}

In the example above: 
\begin{itemize}[leftmargin = *]
\item \colorbox{lightgray}{getFahrenheit()} is called and \colorbox{lightgray}{15} is passed as an argument.
\item The code block inside of \colorbox{lightgray}{getFahrenheit()} calls \colorbox{lightgray}{multiplyByNineFifths()} and passes \colorbox{lightgray}{15} as an argument.
\item \colorbox{lightgray}{multiplyByNineFifths()} takes the argument of \colorbox{lightgray}{15} for the \colorbox{lightgray}{number} parameter.
\item The code block inside of \colorbox{lightgray}{multiplyByNineFifths()} function multiplies \colorbox{lightgray}{15} by \colorbox{lightgray}{(9/5)}, which evaluates to \colorbox{lightgray}{27}.
\item \colorbox{lightgray}{27} is returned back to the function call in \colorbox{lightgray}{getFahrenheit()}.
\item \colorbox{lightgray}{getFahrenheit()} continues to execute. It adds \colorbox{lightgray}{32} to \colorbox{lightgray}{27}, which evaluates to \colorbox{lightgray}{59}.
\item Finally, \colorbox{lightgray}{59} is returned back to the function call \colorbox{lightgray}{getFahrenheit(15)}.
\end{itemize}
We can use functions to section off small bits of logic or tasks, then use them when we need to. Writing helper functions can help take large and difficult tasks and break them into smaller and more manageable tasks.
\newpage
\subsection{Function Expressions}
Another way to define a function is to use a \textit{function expression}. To define a function inside an expression, we can use the \colorbox{lightgray}{function} keyword. In a function expression, the function name is usually omitted. A function with no name is called an \textit{anonymous function}. A function expression is often stored in a variable in order to refer to it. Consider the following function expression:
\begin{figure}[H]
\includegraphics[scale = 0.73]{4_8}
\centering
\end{figure}
To declare a function expression:
\begin{enumerate}[leftmargin = *]
\item Declare a variable to make the variable’s name be the name, or identifier, of your function. Since the release of ES6, it is common practice to use \colorbox{lightgray}{const} as the keyword to declare the variable.
\item Assign as that variable’s value an anonymous function created by using the \colorbox{lightgray}{function} keyword followed by a set of parentheses with possible parameters. Then a set of curly braces that contain the function body.
\end{enumerate}
To invoke a function expression, write the name of the variable in which the function is stored followed by parentheses enclosing any arguments being passed into the function.
\begin{lstlisting}
variableName(argument1, argument2)
\end{lstlisting}
Unlike function declarations, function expressions are not hoisted so they cannot be called before they are defined.

\subsection{Arrow Functions}
ES6 introduced \textit{arrow function syntax}, a shorter way to write functions by using the special “fat arrow” \colorbox{lightgray}{$($$)$ $=$$>$} notation. \\
\newline
Arrow functions remove the need to type out the keyword \colorbox{lightgray}{function} every time you need to create a function. Instead, you first include the parameters inside the \colorbox{lightgray}{( )} and then add an arrow \colorbox{lightgray}{$=$$>$} that points to the function body surrounded in \colorbox{lightgray}{\{ \}} like this:
\begin{lstlisting}
const rectangleArea = (width, height) => {
  let area = width * height;
  return area;
};
\end{lstlisting}
It is important to be familiar with the multiple ways of writing functions because you will come across each of these when reading other JavaScript code.

\subsection{Concise Body Arrow Functions}
JavaScript also provides several ways to refactor arrow function syntax. The most condensed form of the function is known as \textit{concise body}. We will explore a few of these techniques below:
\begin{enumerate}[leftmargin = *]
\item Functions that take only a single parameter do not need that parameter to be enclosed in parentheses. However, if a function takes zero or multiple parameters, parentheses are required.
\begin{figure}[H]
\includegraphics[scale = 0.73]{4_9}
\centering
\end{figure}
\item A function body composed of a single-line block does not need curly braces. Without the curly braces, whatever that line evaluates will be automatically returned. The contents of the block should immediately follow the arrow \colorbox{lightgray}{$=$$>$} and the \colorbox{lightgray}{return} keyword can be removed. This is referred to as \textit{implicit return}.
\end{enumerate}
\begin{figure}[H]
\includegraphics[scale = 0.73]{4_10}
\centering
\end{figure}
So if we have a function: 
\begin{lstlisting}
const squareNum = (num) => {
  return num * num;
};
\end{lstlisting}
We can refactor the function to:
\begin{lstlisting}
const squareNum = num => num * num;
\end{lstlisting}
Notice the following changes:
\begin{itemize}[leftmargin = *]
\item The parentheses around \colorbox{lightgray}{num} have been removed, since it has a single parameter.
\item The curly braces \colorbox{lightgray}{\{ \}} have been removed since the function consists of a single-line block.
\item The \colorbox{lightgray}{return} keyword has been removed since the function consists of a single-line block.
\end{itemize}

\subsection{Review Functions}
In this we covered some important concepts about functions:
\begin{itemize}[leftmargin = *]
\item A \textit{function} is a reusable block of code that groups together a sequence of statements to perform a specific task.
\item A \textit{function declaration}:
\vspace{-4mm}
\begin{figure}[H]
\includegraphics[scale = 0.73]{4_1}
\centering
\end{figure}
\item A parameter is a named variable inside a function’s block which will be assigned the value of the argument passed in when the function is invoked: 
\begin{figure}[H]
\includegraphics[scale = 0.75]{4_4}
\centering
\end{figure}
\item To \textit{call} a function in your code:
\vspace{-2mm}
\begin{figure}[H]
\includegraphics[scale = 0.65]{4_2}
\centering
\end{figure}
\item ES6 introduces new ways of handling arbitrary parameters through \textit{default parameters} which allow us to assign a default value to a parameter in case no argument is passed into the function.
\item To return a value from a function, we use a \textit{return statement}.
\item To define a function using \textit{function} expressions:
\begin{figure}[H]
\includegraphics[scale = 0.73]{4_8}
\centering
\end{figure}
\item To define a function using \textit{arrow function notation}:
\begin{figure}[H]
\includegraphics[scale = 0.8]{4_11}
\centering
\end{figure}
\item Function definition can be made concise using concise arrow notation:
\begin{figure}[H]
\includegraphics[scale = 0.73]{4_10}
\centering
\end{figure}
\end{itemize}
It is good to be aware of the differences between function expressions, arrow functions, and function declarations. As you program more in JavaScript, you will see a wide variety of how these function types are used. 

\newpage
\section{Scope}
\subsection{Scope}
An important idea in programming is \textit{scope}. Scope defines where variables can be accessed or referenced. While some variables can be accessed from anywhere within a program, other variables may only be available in a specific context. \\
\newline
You can think of scope like the view of the night sky from your window. Everyone who lives on the planet Earth is in the global scope of the stars. The stars are accessible \textit{globally}. Meanwhile, if you live in a city, you may see the city skyline or the river. The skyline and river are only accessible \textit{locally} in your city, but you can still see the stars that are available globally. \\
\newline
Over the next few chapters, we will explore how scope relates to variables and learn best practices for variable declaration.

\subsection{Blocks and Scope}
Before we talk more about scope, we first need to talk about \textit{blocks}. We have seen blocks used before in functions and \colorbox{lightgray}{if} statements. A block is the code found inside a set of curly braces \colorbox{lightgray}{\{\}}. Blocks help us group one or more statements together and serve as an important structural marker for our code. \\
\newline
A block of code could be a function, like this: 
\begin{lstlisting}
const logSkyColor = () => {
  let color = "blue"; 
  console.log(color); // blue 
};
\end{lstlisting}
Notice that the function body is actually a block of code. Observe the block in an \colorbox{lightgray}{if} statement:
\begin{lstlisting}
if (dusk) {
  let color = "pink";
  console.log(color); // pink
};
\end{lstlisting}
In the next few chapters, we will see how blocks define the scope of variables.

\subsection{Global Scope}
Scope is the context in which our variables are declared. We think about scope in relation to blocks because variables can exist either outside of or within these blocks. \\
\newline
In \textit{global scope}, variables are declared outside of blocks. These variables are called \textit{global variables}. Because global variables are not bound inside a block, they can be accessed by any code in the program, including code in blocks. \\
\newpage
Let us take a look at an example of global scope:
\begin{lstlisting}
const color = "blue"

const returnSkyColor = () => {
  return color; // blue 
};

console.log(returnSkyColor()); // blue
\end{lstlisting}
\begin{itemize}[leftmargin = *]
\item Even though the \colorbox{lightgray}{color} variable is defined outside of the block, it can be accessed in the function block, giving it global scope.
\item In turn, \colorbox{lightgray}{color} can be accessed within the \colorbox{lightgray}{returnSkyColor} function block.
\end{itemize}

\subsection{Block Scope}
The next context we will cover is \textit{block scope}. When a variable is defined inside a block, it is only accessible to the code within the curly braces \colorbox{lightgray}{\{\}}. We say that variable has \textit{block scope} because it is \textit{only} accessible to the lines of code within that block. \\
\newline
Variables that are declared with block scope are known as \textit{local variables} because they are only available to the code that is part of the same block. \\
\newline
Block scope works like this:
\begin{lstlisting}
const logSkyColor = () => {
  let color = "blue"; 
  console.log(color); // blue 
};

logSkyColor(); // blue 
console.log(color); // ReferenceError
\end{lstlisting}
We will notice:
\begin{itemize}[leftmargin = *]
\item We define a function  \colorbox{lightgray}{logSkyColor()}.
\item Within the function, the  \colorbox{lightgray}{color} variable is only available within the curly braces of the function.
\item If we try to log the same variable outside the function, throws a  \colorbox{lightgray}{ReferenceError}.
\end{itemize}

\subsection{Scope Pollution}
It may seem like a great idea to always make your variables accessible, but having too many global variables can cause problems in a program. \\
\newline
When you declare global variables, they go to the \textit{global namespace}. The global namespace allows the variables to be accessible from anywhere in the program. These variables remain there until the program finishes which means our global namespace can fill up really quickly. \\
\newline
\textit{Scope pollution} is when we have too many global variables that exist in the global namespace, or when we reuse variables across different scopes. Scope pollution makes it difficult to keep track of our different variables and sets us up for potential accidents. For example, globally scoped variables can collide with other variables that are more locally scoped, causing unexpected behavior in our code. \\
\newline
Let us look at an example of scope pollution in practice so we know how to avoid it:
\begin{lstlisting}
let num = 50;

const logNum = () => {
  num = 100; // Take note of this line of code
  console.log(num);
};

logNum(); // Prints 100
console.log(num); // Prints 100
\end{lstlisting}
We will notice:
\begin{itemize}[leftmargin = *]
\item We have a variable \colorbox{lightgray}{num}.
\item Inside the function body of \colorbox{lightgray}{logNum()}, we want to declare a new variable but forgot to use the \colorbox{lightgray}{let} keyword.
\item When we call \colorbox{lightgray}{logNum()}, \colorbox{lightgray}{num} gets reassigned to \colorbox{lightgray}{100}.
\item The reassignment inside \colorbox{lightgray}{logNum()} affects the global variable \colorbox{lightgray}{num}.
\item Even though the reassignment is allowed and we will not get an error, if we decided to use \colorbox{lightgray}{num} later, we’ll unknowingly use the new value of \colorbox{lightgray}{num}.
\end{itemize}
While it is important to know what global scope is, it is best practice to not define variables in the global scope.

\subsection{Practice Good Scoping}
Given the challenges with global variables and scope pollution, we should follow best practices for scoping our variables as tightly as possible using block scope. \\
\newline
Tightly scoping your variables will greatly improve your code in several ways:
\begin{itemize}[leftmargin = *]
\item It will make your code more legible since the blocks will organize your code into discrete sections. 
\item It makes your code more understandable since it clarifies which variables are associated with different parts of the program rather than having to keep track of them line after line! 
\item It is easier to maintain your code, since your code will be modular.
\item It will save memory in your code because it will cease to exist after the block finishes running.
\end{itemize}
Here is another example of how to use block scope, as defined within an \colorbox{lightgray}{if} block:
\begin{lstlisting}
const logSkyColor = () => {
  const dusk = true;
  let color = "blue"; 
  if (dusk) {
    let color = "pink";
    console.log(color); // pink
  }
  console.log(color); // blue 
};

console.log(color); // ReferenceError
\end{lstlisting}
Here, we will notice:
\begin{itemize}[leftmargin = *]
\item We create a variable \colorbox{lightgray}{dusk} inside the \colorbox{lightgray}{logSkyColor()} function.
\item After the \colorbox{lightgray}{if} statement, we define a new code block with the \colorbox{lightgray}{\{\}} braces. Here we assign a new value to the variable \colorbox{lightgray}{color} if the \colorbox{lightgray}{if} statement is truthy.
\item Within the \colorbox{lightgray}{if} block, the \colorbox{lightgray}{color} variable holds the value \colorbox{lightgray}{`pink'}, though outside the \colorbox{lightgray}{if} block, in the function body, the \colorbox{lightgray}{color} variable holds the value \colorbox{lightgray}{`blue'}.
\item While we use block scope, we still pollute our namespace by reusing the same variable name twice. A better practice would be to rename the variable inside the block.
\end{itemize}
Block scope is a powerful tool in JavaScript, since it allows us to define variables with precision, and not pollute the global namespace. If a variable does not need to exist outside a block, then it should not!

\subsection{Review: Scope}
In this chapter, you learned about scope and how it impacts the accessibility of different variables. \\
Let us review the following terms:
\begin{itemize}[leftmargin = *]
\item \textbf{Scope} is the idea in programming that some variables are accessible/inaccessible from other parts of the program.
\item \textbf{Blocks} are statements that exist within curly braces \colorbox{lightgray}{\{\}}.
\item \textbf{Global scope} refers to the context within which variables are accessible to every part of the program.
\item \textbf{Global variables} are variables that exist within global scope.
\item \textbf{Block scope} refers to the context within which variables that are accessible only within the block they are defined.
\item \textbf{Local variables} are variables that exist within block scope.
\item \textbf{Global namespace} is the space in our code that contains globally scoped information.
\item \textbf{Scope pollution} is when too many variables exist in a namespace or variable names are reused.
\end{itemize}




\end{document}
